\begin{abstract}
\addcontentsline{toc}{chapter}{Abstract}
In recent years ground-based interferometers such as LIGO and Virgo opened the high-frequency window of the rapidly evolving gravitational wave astronomy.
Future space-based detectors, in particular the \textit{Laser Interferometer Space Antenna} (LISA), will extend gravitational wave detection into the low-frequency band, where Galactic compact binaries such as double white dwarfs (DWDs) are expected to shape LISA's sensitivity curve significantly;
the aim of this work is to investigate the expected gravitational wave background produced by \textit{extragalactic} DWDs in the local Universe, employing the \textit{Compact Object Synthesis and Monte Carlo Investigation Code} (\textbf{COSMIC}) to generate synthetic populations representing the galaxies included in the observational survey \textbf{Gravitational Wave Galaxy Catalog} (GWGC).
Each simulated binary contributes to the total gravitational wave strain, which is then combined across frequency bins and compared against the LISA sensitivity curve.
The results show that the total simulated extragalactic signal lies approximately one order of magnitude below LISA’s detection threshold, indicating that the noise from the unresolved sources will not produce a measurable imprint on the sensitivity curve.
Future developments could include the use of more advanced population synthesis codes and different prescriptions for binary evolution, or the extension of the analysis to larger observational galaxy catalogs and cosmological volumes.
\end{abstract}
