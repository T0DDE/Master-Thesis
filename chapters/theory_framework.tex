\chapter{Gravitational Waves Theory}
After considering, in 1905, the problem of the apparently \textit{instantaneous} propagation of light, with the theory of Special Relativity, in 1916 Albert Einstein considered the problem of the apparently \textit{instantaneous} propagation of gravity through \textit{long distances}, in his theory of General Relativity.
Einstein showed that long-distance interaction arises from the deformation of spacetime caused by massive objects.
Hence, in the "static case", the deviated motion apparently caused by the interaction between two distant masses really is, in fact, a manifestation of spacetime curvature nearby, generated by the presence of the two objects.
The "static case" just depicted, though, treats the curvature as if it had always been there, and doesn't take into account of any variation in the masses, positions or velocities of the two objects, that would induce an evolution to the curvature itself.
In truth, after a change in the mass-energy distribution, the corresponding curvature variation requires its time to reach far distances, and a fascinating prediction of General Relativity is that it propagates in the form of a wave, that travels at the speed of light.


\section{Gravitational Waves as Perturbations}
The tipical approach to the study of gravitational waves is to derive them as small perturbations of the background from the Einstein's equations: 
\begin{equation}
    G_{\mu\nu} = R_{\mu\nu} - \frac{1}{2}g_{\mu\nu}R,
\end{equation}
which can be conveniently written as 
\begin{equation}
    R_{\mu\nu}=\frac{8\pi G}{c^4}\left(T_{\mu\nu} - \frac{1}{2}g_{\mu\nu}T\right),
    \label{eq: Einstein equations rewritten}
\end{equation}
where $G_{\mu\nu}$ is the Einstein tensor, $R_{\mu\nu}$ is the Ricci tensor, $R$ is the Ricci scalar, $g_{\mu\nu}$ is the metric of spacetime, and $T_{\mu\nu}$ is the stress-energy tensor.
As a background solution we can consider the flat spacetime described by the metric $\eta_{\mu\nu}$, to which the perturbation term appears as a fluctuation in the metric $|h_{\mu\nu}|\ll|\eta_{\mu\nu}|$, known as \textit{weak field} approximation. Thus, the perturbed spacetime can be written as:
\[
    g_{\mu\nu} = \eta_{\mu\nu} + h_{\mu\nu}, \hspace{5mm}  |h_{\mu\nu}|\ll|\eta_{\mu\nu}|.
\] 
With this metric, the equations \ref{eq: Einstein equations rewritten} becomes
\[
    \{\Box_F h_{\mu\nu} - \left[\frac{\partial^2}{\partial x^\lambda\partial x^\mu}h^\lambda_\nu + \frac{\partial^2}{\partial x^\lambda\partial x^\nu}h^\lambda_\mu + \frac{\partial^2}{\partial x^\nu\partial x^\mu}h^\lambda_\lambda\right]\} = -\frac{16\pi G}{c^4}\left(T_{\mu\nu} - \frac{1}{2}\eta_{\mu\nu}T\right).
\]
Now, by requiring that the \textit{weak-field} approximation remains satisfied for infinitesimal diffeomorphisms, and by choosing a coordynate system in which the \textit{harmonic gauge condition}\footnote{It is an arbitrary coordinate condition which makes it possible to solve the Einstein field equations. It can be found by requiring that the linearized Einstein equations satisfy the D'Alambert equation.} is satisfied, we can find that, up to first order in $h_{\mu\nu}$, the harmonic gauge condition is equivalent to 
\begin{equation}
    \frac{\partial}{\partial x^\mu}h^\mu_\rho = \frac{1}{2}\frac
    {\partial}{\partial x^\rho}h, \hspace{5mm} h=\eta^{\mu\nu}h_{\mu\nu}\equiv h^\nu_\nu,
\end{equation}
and after defining the tensor\footnote{Also known as \textit{trace-reversed} perturbation tensor, since $\bar{h}=\eta^{\mu\nu}\bar{h}_{\mu\nu} = -h$.}
\[
    \bar{h}_{\mu\nu} \equiv h_{\nu\mu} - \frac{1}{2}\eta_{\mu\nu}h,
\]
we can finally write the linearized Einstein equations as

\[
    \left\{
\begin{aligned}
    \Box\, \bar{h}_{\mu\nu} &= -\frac{16\pi G}{c^4} T_{\mu\nu}, 
    \notag\\
    \partial^\mu \bar{h}_{\mu\nu} &= 0.
\end{aligned}
\right.
\]
This form, and its twin with $T_{\mu\nu}=0$, where the first equation becomes the D'Alambert equation, are relevant because they show that \textit{a perturbation of a flat spacetime propagates as a wave travelling at the speed of light}.


\subsection{Harmonic gauge}
\subsection{The TT gauge}


\section{Motion and geodesics}
\subsection{The motion deviation}
\subsection{The geodesic deviation}

\section{The Quadrupole Approximation}
\subsection{The weak-field, slow-motion approximation}
\subsection{The quadrupole formula}
\subsection{Transform to the TT gauge}

\section{Gravitational waves from a binary system}
\subsection{General solution for circular orbits}
Up to 13.86/7 at page 255 of the book.

\section{Energy carried by a gravitational wave}
\subsection{Stress-energy pseudo-tensor}
\subsection{Gravitational wave luminosity}

\section{Evolution of a compact binary system}
\subsection{Signal from inspiralling compact objects}
Here we get to the actual amplitude we used, and the parameters involved.

\section{LISA}
\section{interferometers}



- Instrument description (what is an interferometer, why in space, how it will be made, orbit)
- frequency band - what will it see?
- frequency resolution
- sensibility curve and ASD meaning
- Why WD choice in particular?

