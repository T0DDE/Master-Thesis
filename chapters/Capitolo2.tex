\chapter{Gravitational Waves Theory}
After considering, in 1905, the problem of the apparently \textit{instantaneous} propagation of light, with the theory of Special Relativity, in 1916 Albert Einstein considered the problem of the apparently \textit{instantaneous} propagation of gravity through \textit{long distances}, in his theory of General Relativity.
Einstein showed that long-distance interaction arises from the deformation of spacetime caused by massive objects.
Hence, in the "static case", the deviated motion apparently caused by the interaction between two distant masses really is, in fact, a manifestation of spacetime curvature nearby, generated by the presence of the two objects.
The "static case" just depicted, though, treats the curvature as if it had always been there, and doesn't take into account of any variation in the masses, positions or velocities of the two objects, that would induce an evolution to the curvature itself.
In truth, after a change in the mass-energy distribution, the corresponding curvature variation requires its time to reach far distances, and a fascinating prediction of General Relativity is that it propagates in the form of a wave, that travels at the speed of light.



\section{The Flat Spacetime}

\section{Gravitational Waves as Perturbations}
\subsection{Harmonic gauge}
\subsection{The TT gauge}


\section{Motion and geodesics}
\subsection{The motion deviation}
\subsection{The geodesic deviation}

\section{The Quadrupole Approximation}
\subsection{The weak-field, slow-motion approximation}
\subsection{The quadrupole formula}
\subsection{Transform to the TT gauge}

\section{Gravitational waves from a binary system}
\subsection{General solution for circular orbits}
Up to 13.86/7 at page 255 of the book.

\section{Energy carried by a gravitational wave}
\subsection{Stress-energy pseudo-tensor}
\subsection{Gravitational wave luminosity}

\section{Evolution of a compact binary system}
\subsection{Signal from inspiralling compact objects}
Here we get to the actual amplitude we used, and the parameters involved.
\subsection{ASD and multiple sources}
