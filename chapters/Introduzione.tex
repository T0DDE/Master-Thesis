\chapter*{Introduction}
\addcontentsline{toc}{chapter}{Introduction}
In the context of the gravitational waves study, there are mainly two possible paths: the first is the \textbf{analysis of existing data} from experiments like Ligo, Virgo and Kagra, with the goal of detecting and characterizing the observable sources within their relative frequency bands;
the second approach is the \textbf{forecasting approach}, which aims to characterize future experiments in order to better understand what types of sources they could detect and how well.
\vspace{2mm}\\
One of the most important future detectors is the European \textit{Laser Interferometer Space Antenna} (LISA), the first space-based gravitational wave detector. 
LISA will be arranged as an equilateral triangle with 2.5 million kilometers long arms, placed in a heliocentric orbit, and will operate within the frequency range from approximately $0.1\mathrm{mHz}$ to $1\mathrm{Hz}$. 
Among the typical sources that emit gravitational wave signal in this range are the \textbf{compact binaries}, and in our particular case we are interested in \textbf{double white dwarf binaries} (\textbf{WDBs}).
\vspace{2mm}\\
The goal of this work is to estimate the gravitational wave background produced by the extragalactic WDBs in the local universe, by using the \textbf{COSMIC} code to generate synthetic astrophysical populations to represent the galaxies listed in the \textbf{Gravitational Wave Galaxy Catalog} (\textbf{GWGC}).\vspace{2mm}
\\  
In \textbf{Chapter 1} we will introduce the theoretical foundations of gravitational waves, derive the amplitude of the signal generated by a binary system, and define the most important parameters. Finally, we discuss how to combine the signals from multiple sources, to find a cumulative background.\vspace{2mm}\\
In \textbf{Chapter 2} we will give a brief overview of how gravitational wave detectors work, trying to better understand what kinds of sources LISA will be able to see, what resolution and what sensitivity it will have and why in particular we are interested in WDBs. \vspace{2mm}\\
In \textbf{Chapter 3} we introduce the concept of stellar population synthesis and the code COSMIC used for this purpose. We will introduce its features, main parameters, and pipeline, and explain how it is used to generate full-size astrophysical populations of WDBs.\vspace{2mm}\\ 
In \textbf{Chapter 4} we introduce the GWGC, list the key information it provides and explain how we move from there to infer the remaining parameters that we need.\vspace{2mm}\\
In \textbf{Chapter 5} we use the obtained information to compute the total gravitational wave signal summing the contribution from all the simulated sources, taking into account their spatial distribution, LISA's frequency resolution, and the \textit{zone of avoidance} caused by the milky way.\vspace{2mm}\\
In \textbf{Chapter 6} we plot the total resulting signal on the LISA sensitivity curve, and discuss the results and their possible implications.\vspace{2mm}\\
Finally, in \textbf{Chapter 7} we will draw some conclusions from the work as a whole, discussing its limitations, assumptions and its possible extensions and follow-ups.
