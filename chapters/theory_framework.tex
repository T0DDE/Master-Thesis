\chapter{Theoretical Framework}


\section{Gravitational Waves Theory}
After considering, in 1905, the problem of the apparently \textit{instantaneous} propagation of light, with the theory of Special Relativity, in 1916 Albert Einstein considered the problem of the apparently \textit{instantaneous} propagation of gravity through \textit{long distances}, in his theory of General Relativity.
Einstein showed that long-distance interaction arises from the deformation of space time caused by massive objects.
Hence, in the "static case", the deviated motion apparently caused by the interaction between two distant masses really is, in fact, a manifestation of space time curvature nearby, generated by the presence of the two objects.
The "static case" just depicted, though, treats the curvature as if it had always been there, and doesn't take into account of any variation in the masses, positions or velocities of the two objects, that would induce an evolution to the curvature itself.
In truth, after a change in the mass-energy distribution, the corresponding curvature variation requires its time to reach far distances, and a fascinating prediction of General Relativity is that it propagates in the form of a wave, that travels at the speed of light.
\subsection{Gravitational Waves as Perturbations}
The typical approach to the study of gravitational waves is to derive them as small perturbations of the background from the Einstein's equations: 
\begin{equation}
    G_{\mu\nu} = R_{\mu\nu} - \frac{1}{2}g_{\mu\nu}R,
\end{equation}
which can be conveniently written as 
\begin{equation}
    R_{\mu\nu}=\frac{8\pi G}{c^4}\left(T_{\mu\nu} - \frac{1}{2}g_{\mu\nu}T\right),
    \label{eq: Einstein equations rewritten}
\end{equation}
where $G_{\mu\nu}$ is the Einstein tensor, $R_{\mu\nu}$ is the Ricci tensor, $R$ is the Ricci scalar, $g_{\mu\nu}$ is the metric of space time, and $T_{\mu\nu}$ is the stress-energy tensor.
As a background solution we can consider the flat space time described by the metric $\eta_{\mu\nu}$, to which the perturbation term appears as a fluctuation in the metric $|h_{\mu\nu}|\ll|\eta_{\mu\nu}|$, known as \textit{weak field} approximation. Thus, the perturbed space time can be written as:
\[
    g_{\mu\nu} = \eta_{\mu\nu} + h_{\mu\nu}, \hspace{5mm}  |h_{\mu\nu}|\ll|\eta_{\mu\nu}|.
\] 
With this metric, the equations \ref{eq: Einstein equations rewritten} becomes
\[
    \{\square_F h_{\mu\nu} - \left[\frac{\partial^2}{\partial x^\lambda\partial x^\mu}h^\lambda_\nu + \frac{\partial^2}{\partial x^\lambda\partial x^\nu}h^\lambda_\mu + \frac{\partial^2}{\partial x^\nu\partial x^\mu}h^\lambda_\lambda\right]\} = -\frac{16\pi G}{c^4}\left(T_{\mu\nu} - \frac{1}{2}\eta_{\mu\nu}T\right).
\]
Now, by requiring that the \textit{weak-field} approximation remains satisfied for infinitesimal diffeomorphisms, and by choosing a coordinate system in which the \textit{harmonic gauge condition}\footnote{It is an arbitrary coordinate condition which makes it possible to solve the Einstein field equations. It can be found by requiring that the linearized Einstein equations satisfy the D'Alambert equation.}, defined as 
\begin{equation}
    \Gamma^\lambda = g^{\mu\nu}\Gamma^\lambda_{\mu\nu} = 0,
    \label{eq: Harmonic gauge definition}
\end{equation}
where $\Gamma^\lambda_{\mu\nu}$ are the \textit{affine connections}, is satisfied, we can find that, up to first order in $h_{\mu\nu}$, the harmonic gauge condition is equivalent to 
\begin{equation}
    \frac{\partial}{\partial x^\mu}h^\mu_\rho = \frac{1}{2}\frac
    {\partial}{\partial x^\rho}h, \hspace{5mm} h=\eta^{\mu\nu}h_{\mu\nu}\equiv h^\nu_\nu.
\end{equation}
After defining the \textit{trace-reversed}\footnote{The name comes by noting that $\bar{h}=\eta^{\mu\nu}\bar{h}_{\mu\nu} = -h$.} tensor as
\begin{equation*}
    \bar{h}_{\mu\nu} \equiv h_{\nu\mu} - \frac{1}{2}\eta_{\mu\nu}h,
\end{equation*}
we can finally write the linearized Einstein equations as
\begin{equation}
    \left\{
        \begin{aligned}
            \square\bar{h}_{\mu\nu} &= -\frac{16\pi G}{c^4} T_{\mu\nu}, 
            \notag\\
            \partial^\mu \bar{h}_{\mu\nu} &= 0.
        \end{aligned}
    \right.
    \label{eq: Einstein equations as wave equations}
\end{equation}
This form, and its twin with $T_{\mu\nu}=0$, where the first equation becomes the D'Alambert equation, are relevant because they show that \textit{a perturbation of a flat space time propagates as a wave travelling at the speed of light}.
As in Electrodynamics, the solution of~\eqref{eq: Einstein equations as wave equations} can be written in terms of \textit{retarded potentials}:
\begin{equation}
    \bar{h}_{\mu\nu}(t,\mathbf{x}) = \frac{4G}{c^4}\int_V \frac{T_{\mu\nu}(t - \frac{|\mathbf{x} - \mathbf{x}'|}{c}, \mathbf{x}')}{|\mathbf{x} - \mathbf{x}'|}d^3x',
    \label{eq: solutions with retarded potentials}
\end{equation}
where V is the three dimensional source volume, $\mathbf{x}'$ is the distance of an element of the emitting source from the origin of a frame centered in the same point of the source, $\mathbf{x}$ is the distance between source and observer.
It can be proved that the solutions in \eqref{eq: solutions with retarded potentials} automatically satisfy the harmonic gauge condition in the second equation of the \eqref{eq: Einstein equations as wave equations}.


\subsubsection{Harmonic gauge}
It is important to notice that if the \textit{harmonic gauge} condition is not satisfied in a reference frame, a new frame in which it is can always be found, by making an infinitesimal coordinate transformation
\begin{equation}
    x^{\lambda'} = x^\lambda + \epsilon^\lambda,
    \label{eq: infinitesimal coordinate transformation}
\end{equation}
provded that $\epsilon^\lambda$ satisfies the following equation:
\[
    \square_F\epsilon_\rho = \frac{\partial h^\beta_\rho}{\partial x^\beta} - \frac{1}{2} \frac{\partial h}{\partial x^\rho} = 0\hspace{1mm}.
\]
This is an inhomogeneous wave equation that can be solved to find the components $\epsilon_\alpha$, which identify the coordinate system in which the harmonic gauge condition is satisfied.
Notice though, that the harmonic condition in \eqref{eq: Harmonic gauge definition} does not determine the gauge uniquely, but instead leaves some more gauge freedom to be used.

\subsection{The physical wave}
As we have seen, perturbations of flat space-time satisfy a wave equation and a harmonic gauge condition, as in \eqref{eq: Einstein equations as wave equations}.
The general solutions of these wave equation is a linear superposition of monochromatic plane waves, with a polarization tensor (or wave amplitude) $A_{\mu\nu}$ and a wave four-vector $\vec{k}$, such as
\[
    \square_F \bar{h}{\mu\nu} = -A_{\mu\nu}\eta^{\alpha\beta} k_\alpha k_\beta e^{ik_\gamma x^\gamma} = 0.
\]
Thus, neglecting the trivial solution $A_{\mu\nu} = 0$, gives 
\[
    \eta^{\alpha\beta} k_\alpha k_\beta = 0,
\]
which means that $\vec{k}$ is a null vector.
If we also consider the harmonic gauge, we find a condition that imposes the orthogonality  of the wave four-vector to the polarization tensor,
\[
    k_\mu A^\mu_\nu = 0.
\]
The \textit{wavefronts}, i.e. the spatial surfaces where $\bar{h}_{\mu\nu}= const.$ are the planes where $k_ix^i=const.$
Conventionally, $k^0$ is referred to as $\frac{\omega}{c}$, where $\omega$ is the frequency, thus
\[
    \vec{k}_0 = \left(\frac{\omega}{c},\mathbf{k}\right),
\]
where $\mathbf{k}$ is the wave three-vector orthogonal to the wavefront, and is related to the \textit{wavelenght} by $|\mathbf{k}| = 2\pi/\lambda$.
Notice that, since the wave four-vector $\vec{k}$ is a null vector, it follows that
\[
    -(k^0)^2 + |\mathbf{k}| = 0 \to \omega = ck_0 = c|\mathbf{k}|,
\]
which gives the dispersion relation for a wave moving at the speed of light.


\subsubsection{The TT gauge}
In a one dimension case, the wave equation can be written as
\[
    \left( -\frac{1}{c^2} \frac{\partial^2}{\partial t^2}  + \frac{\partial^2}{\partial x^2}\right)\bar{h}^\mu_\nu = 0,
\]
which generally has, as solution, an arbitrary function of $t\pm \frac{x}{c}$.
If we consider, for example, a progressive wave $\bar{h}^\mu_\nu [f(t,x)]$, where $f(t,x) = t- \frac{x}{c}$, and apply the harmonic gauge condition, focusing only in the time-depndent part of the solution, we find the \textit{first four} conditions:
\begin{equation}
    \bar{h}^t_t = \bar{h}^x_t, \hspace{5mm} \bar{h}^t_x = \bar{h}^x_x, \hspace{5mm} \bar{h}^t_y = \bar{h}^x_y, \hspace{5mm} \bar{h}^t_z = \bar{h}^x_z.
    \label{eq: first four conditions}
\end{equation}
As we have already discussed, there still is the freedom of making an infinitesimal coordinate change, and by requiring that the harmonic gauge remains satisfied in the new coordinates generates \textit{four more conditions}:
\begin{equation}
    \bar{h}^t_x = \bar{h}^t_y = \bar{h}^t_z = \bar{h}^y_y + \bar{h}^z_z = 0,
    \label{eq: second four conditions}
\end{equation}
and with the~\eqref{eq: first four conditions} follows
\[
     \bar{h}^x_x = \bar{h}^x_y = \bar{h}^x_z = \bar{h}^t_t = 0.
\]
The only non zero components are $\bar{h}^z_y$ and $\bar{h}^y_y - \bar{h}^z_z$, which cannot be set to zero because now we have completely used all the gauge freedom.
It can be shown that, from all the above conditions, it follows that $\bar{h} \equiv h$, i.e. in this gauge the wave results to be \textit{traceless}.
Thus, a plane gravitational wave propagating along the $x$-axis is characterized by only two non-zero functions $h_{zy}$ and $h_{yy}=-h_{zz}$:
\begin{equation}
    h^{TT}_{\mu\nu} =  
    \begin{pmatrix}
0 & 0 & 0 & 0 \\
0 & 0 & 0 & 0 \\
0 & 0 & h_{yy} & h_{yz} \\
0 & 0 & h_{yz} & -h_{yy}
\end{pmatrix}.
    \label{eq: h as a matrix in x case}
\end{equation}
In conclusion, the gravitational wave only has \textbf{two physical degrees of freedom} which correspond to two polarization states.
This gauge is called \textbf{TT gauge} because of the \textit{transverse traceless} nature: $h_{\mu\nu}$ is traceless, thus $h=0$, and transverse, since the components of $h_{\mu\nu}$ along the direction of propagation are null (in this case $h_{\mux}=0$).






\subsection{Motion and geodesics}
\subsubsection{The motion deviation}
\subsubsection{The geodesic deviation}

\subsection{The Quadrupole Approximation}
\subsubsection{The weak-field, slow-motion approximation}
\subsubsection{The quadrupole formula}
\subsubsection{Transform to the TT gauge}

\subsection{Gravitational waves from a binary system}
\subsubsection{General solution for circular orbits}
Up to 13.86/7 at page 255 of the book.

\subsection{Energy carried by a gravitational wave}
\subsubsection{Stress-energy pseudo-tensor}
\subsubsection{Gravitational wave luminosity}

\subsection{Evolution of a compact binary system}
\subsubsection{Signal from inspiralling compact objects}
Here we get to the actual amplitude we used, and the parameters involved.

\section{White dwarfs and galaxies}
\subsection{White dwarfs}
\subsection{Galaxies}
\subsubsection{Morphological types}
Here I introduce Hubble types, explaining general physical differences between them.
Later I will introduce the T value notation and how to translate in the Hubble sequence types. 

\section{LISA}
\section{interferometers}



- Instrument description (what is an interferometer, why in space, how it will be made, orbit)
- frequency band - what will it see?
- frequency resolution
- sensibility curve and ASD meaning
- Why WD choice in particular?

