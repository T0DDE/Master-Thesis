\chapter*{Introduction}
\addcontentsline{toc}{chapter}{Introduction}
In the context of the gravitational waves study, there are mainly two possible paths: the first is the \textbf{analysis of existing data} from experiments like Ligo, Virgo and Kagra, with the purpose of detecting and characterizing the observable sources within their relative frequency bands;
the second is the \textbf{forecasting approach}, which aims to characterize future experiments in order to better understand what types of sources they could be able to detect and how well.
\vspace{2mm}\\
One of the most important future detectors is the European \textit{Laser Interferometer Space Antenna} (LISA), the first space-based gravitational wave detector. 
LISA will be arranged as an equilateral triangle with 2.5 million kilometers long arms, placed in a heliocentric orbit, and will operate within the frequency range from approximately $0.1\mathrm{mHz}$ to $1\mathrm{Hz}$. 
Among the typical sources that emit gravitational wave signal in this range are the \textbf{compact binaries}, and in our particular case we are interested in \textbf{Double White Dwarfs} (\textbf{DWDs}).
\vspace{2mm}\\
Many studies have been made to explore the effects of different type of sources on the LISA's sensitivity curve.
For example, we know that the unresolved binary Galactic systems produce a confusion gravitational wave foreground that shapes noticeably the curve close to its lowest threshold.
While the Galactic contribution has been extensively investigated, less attention has been devoted to the possible extragalactic binaries backgrounds.
Therefore, as a natural continuation, the purpose of this work is to estimate the gravitational wave background produced by the \textit{extragalactic} DWDs in the local universe, using the \textbf{COSMIC} code to generate synthetic astrophysical populations and the galaxies listed in the \textbf{Gravitational Wave Galaxy Catalog} (\textbf{GWGC}), an observational survey, as a reference.
The potential impact of these sources in the tentative to characterize the expected LISA sensitivity and to avoid misinterpreting unresolved signals, are the factors that make this research so important.
\vspace{2mm}
\\  
In \textbf{Theoretical Framework} we introduce the theoretical foundations of gravitational waves, starting from the linearized Einstein field equations, to derive the amplitude of the signal generated by an inspiralling binary system, and define the most important parameters.
We also give a broad theoretical astrophysical framework, briefly discussing the birth and evolution of double white dwarfs, both as single objects, and in binary systems, and the galactic environments that will host them.
We then give a brief overview of how gravitational wave detectors work, trying to better understand what kinds of sources LISA will be able to see, what resolution and what sensitivity it will have and why in particular we are interested in DWDs.
\vspace{2mm}\\
In \textbf{Methods and Data} we introduce the concept of stellar population synthesis and the code COSMIC used for this purpose.
We present its features, main parameters, and pipeline, and explain how it is used to generate full-size astrophysical populations of DWDs.
In order to do this, we introduce the Gravitational Wave Galaxy Catalog, an observational survey that lists some key information about the galaxies, and explain how we used them to infer the remaining parameters.
We then explain the procedure to generate the actual fixed populations, scale them to astrophysical full-size galaxies, and compute their gravitational wave signal. 
Moreover, we discuss how to take care of the lack of galaxies due to the catalog's incompleteness and the Milky Way coverage in the Zone Of Avoidance.
Finally, we discuss the LISA frequency resolution problem, and how it affects the way we must process the gravitational waves signals simulated.\vspace{2mm}\\
In \textbf{Results and Discussion} we present the results we found with the obtained information and the procedure we described, by plotting the total resulting signal on the LISA's sensitivity curve.
We then discuss what we found, and the possible implications. \vspace{2mm}\\
Finally, in the \textbf{Conclusions and Future Perspectives} we recap the work as a whole, discussing its limitations, assumptions and its possible extensions and follow-ups.
