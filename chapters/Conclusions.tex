\chapter*{Conclusions and Future Perspectives}
\addcontentsline{toc}{chapter}{Conclusions and Future Perspectives}
In this thesis we investigated the contribution of extragalactic double white dwarf (DWD) binaries to the low-frequency gravitational wave (GW) background, with a focus on their potential detectability by the space-based interferometer LISA.  
We began by discussing the theoretical gravitational wave framework, and the astrophysical elements that are relevant to this problem, including a brief description of the formation and evolution of white dwarfs, both as isolated stars and in binary systems, and the galactic environments in which they reside and evolve. 
We then discussed the fundamentals of interferometric detection, and the specific capabilities of LISA as a space-based detector.  
On this theoretical basis, we developed a computational pipeline to simulate synthetic populations of extragalactic DWD binaries using the population synthesis code COSMIC. 
We then scaled them according to the stellar masses of host galaxies, using the astrophysical observations galaxy survey "\textit{Gravitational Wave Galaxy Catalog}" as reference, and computed their GW strain and power spectral density. 
The simulated signals were finally binned according to LISA’s frequency resolution and compared with its sensitivity curve.
The main result of this study is that the predicted gravitational wave signal from the extragalactic DWD population lies approximately one order of magnitude below LISA’s detection threshold\footnote{For the LISA curve presented in \cite{Robson_2019}} across the considered frequency band.  
Therefore, although, as we know, DWD binaries are expected to constitute a dominant foreground from Galactic sources, able to shape LISA's curve substantially, the extragalactic component is not expected to measurably affect it at all.
While the present analysis points towards a negligible extragalactic contribution for LISA, it provides a quantitative baseline that can be refined in several directions.  
For example, future work could incorporate more detailed galaxy catalogs with redshift-dependent star formation histories and metallicity evolution, and that take in account more precisely the missing sources blocked from our Galaxy in the \textit{Zone Of Avoidance}.
Similarly, improving the binary population synthesis models, for example by including different prescriptions also on other parameters, and not only focusing on the metallicity, like mass transfer prescriptions or alternative common-envelope outcomes.
Moreover, since population synthesis is a very active and rapidly evolving research field, testing alternative codes, possibly based on different Single Star and Binary Stars evolution models, could help quantify the model dependence of our predictions and reveal systematic uncertainties.
Finally, although the extragalactic DWD background may be undetectable for LISA, this work shows how future space-based detectors with longer baselines or improved sensitivity in the sub-millihertz regime could, in principle, probe this population, offering a complementary view to Galactic studies and providing constraints on the binary evolution across cosmic time.
In summary, this work has explored a previously the extragalactic DWDs component of the GW landscape, reinforcing the understanding that the gravitational wave sky at low frequencies will be dominated by Galactic sources, but also highlighting that the extragalactic background, though faint, remains a scientifically interesting target for the next generation of detectors.

